\documentclass{article}
\usepackage[utf8x]{inputenc}
\usepackage[english,russian]{babel}
\usepackage{graphicx}
\usepackage{todonotes}
\usepackage{hyperref}
\begin{document}
\title{Отчёт о проделанной работе (август - сентябрь 2014)}
\author{\textbf{м.н.с. Голубев В.И.} \\ Лаборатория прикладной вычислительной геофизики МФТИ}
\maketitle

По итогам последней встречи в МФТИ был составлен следующий план работ:
\begin{enumerate}
\item \label{it_wave_migration} Реализовать миграцию по формуле Кирхгофа. (Голубев)
\item \label{it_ray} Расписать формулу Рэлея для случая дискретного представления. (Голубев)
\item \label{it_vector} Расписать численный аналог формулы Кирхгофа для векторного волнового уравнения. (Векторный аналог формулы Рэлея). (Голубев, Фаворская)
\item Расписать численный аналог формулы Кирхгофа для уравнения Ламе. (Голубев, Фаворская)
\item Выписать в явном виде линейный оператор прямой задачи в приближении Борна. (Фаворская, Хохлов, Дашкевич)
\item Выписать в явном виде прямой оператор моделирования в finite-difference approximation. (Петров, Хохлов)
\item Реализовать миграцию на основе сопряженного оператора для акустического уравнения. (Дашкевич, Голубев) 
\end{enumerate}

Существенный задел по задаче \ref{it_wave_migration} был сделан на предыдущем этапе.
Таким образом, в настоящий момент разработана программа, производящая миграцию для скалярного волнового уравнения
при расстановке источников-приёмников типа zero-offset.
Она протестирована на синтетических примерах, выполнено сравнение с комплексом Madagascar.
При этом использовалось решение задачи \ref{it_ray}.
Алёной были получены формулы, описывающие решение задачи \ref{it_vector}.
Отличительной чертой является тот факт, что на каждую компоненту решение выглядит в виде, представленном
в задаче \ref{it_ray}.
Таким образом, для программной реализации достаточно просто увеличить число массивов с 1 до 3.
Однако, поскольку не ясно, какое именно физическое явление описывает векторное волновое уравнение, код не расширялся
на этот случай.
Алёной также были выписаны в явном виде формулы для расчёта прямой упругой задачи.
В настоящий момент идёт их реализация в программном коде (Голубев).

Класс задач, включающий в себя решение задачи миграции на основе сопряжённого оператора, должен был быть выполнен
при участии Антона Дашкевича, однако, продвижений в данном направлении не произошло, кроме того исполнитель был
переведён на другие задачи.

\end{document}
