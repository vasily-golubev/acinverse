\documentclass{article}
\usepackage[utf8x]{inputenc}
\usepackage[english,russian]{babel}
\begin{document}
\title{Применение формулы Кирхгофа для задачи миграции}
\author{\textbf{Голубев В.И.} \\ Лаборатория прикладной вычислительной геофизики МФТИ}
\maketitle
Согласно выводу (уравнени 13.208), приведённому в книге Жданова, распределение давления для
задачи акустики, показывающее также и положение отражающих границ,
может быть найдено на основе формулы Рэлея для случая горизонтальной плоскости
наблюдения:
\begin{equation}
\label{rayleigh_migration}
U^m(\vec{r'},0) = -\frac{1}{2\pi}\frac{\partial}{\partial z'}
	\int_S \frac{U(\vec{r},\frac{2|\vec{r'}-\vec{r}|}{c})}{|\vec{r'}-\vec{r}|}ds,
\end{equation}
где $U$ - скалярное поле давления, $\vec{r}$ - координата на поверхности, в которой записывается
сейсмограмма, $\vec{r'}$ - координата в объёме, где ищется мигрированное изображение, $c$ - скорость распространения возмущения, стоящая в волновом уравнении (уравнение 13.54):
\begin{equation}
\label{wave_equation}
\nabla^2P(\vec{r},t) - \frac{1}{c^2(\vec{r})}\frac{\partial^2}{\partial t^2}
	P(\vec{r},t) = - F^e(\vec{r},t).
\end{equation}
\end{document}
